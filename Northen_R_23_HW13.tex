%%%%%%%%%%%%%%%%%%%%%%%%%%%%%%%%%%%%%%%%%%%%%%%%%%%%%%%%%%%%
%%%%%%%%%%%%%%%%%%%%%%%%%%%%%%%%%%%%%%%%%%%%%%%%%%%%%%%%%%%%
%%%%%%%%%%%%%%%%%%%%%%%%%%%%%%%%%%%%%%%%%%%%%%%%%%%%%%%%%%%%
%%%%%%%%%%%%%%%%%%%%%%%%%%%%%%%%%%%%%%%%%%%%%%%%%%%%%%%%%%%%
%%%%%%%%%%%%%%%%%%%%%%%%%%%%%%%%%%%%%%%%%%%%%%%%%%%%%%%%%%%%
\documentclass[12pt]{article}
\usepackage{epsfig}
\usepackage{times}
\usepackage{amsmath}
\usepackage{xcolor}
\usepackage{fancyhdr}
\renewcommand{\topfraction}{1.0}
\renewcommand{\bottomfraction}{1.0}
\renewcommand{\textfraction}{0.0}
\setlength {\textwidth}{6.6in}
\hoffset=-1.0in
\oddsidemargin=1.00in
\marginparsep=0.0in
\marginparwidth=0.0in                                                                               
\setlength {\textheight}{9.0in}
\voffset=-1.00in
\topmargin=1.0in
\headheight=0.0in
\headsep=0.00in
\footskip=0.50in                                        
\fancyfoot{}
\pagestyle{fancy}
\renewcommand{\headrulewidth}{0pt}
\fancyfoot[R]{31}
\fancyfoot[L]{Latex Example}
\begin{document}
\def\pos{\medskip\quad}
\def\subpos{\smallskip \qquad}
\newfont{\nice}{cmr12 scaled 1250}
\newfont{\name}{cmr12 scaled 1080}
\newfont{\swell}{cmbx12 scaled 800}
%%%%%%%%%%%%%%%%%%%%%%%%%%%%%%%%%%%%%%%%%%%%%%%%%%%%%%%%%%%%
%     DO NOT CHANGE ANYTHING ABOVE THIS LINE
%%%%%%%%%%%%%%%%%%%%%%%%%%%%%%%%%%%%%%%%%%%%%%%%%%%%%%%%%%%%
%     DO NOT CHANGE ANYTHING ABOVE THIS LINE
%%%%%%%%%%%%%%%%%%%%%%%%%%%%%%%%%%%%%%%%%%%%%%%%%%%%%%%%%%%%
%     DO NOT CHANGE ANYTHING ABOVE THIS LINE
%%%%%%%%%%%%%%%%%%%%%%%%%%%%%%%%%%%%%%%%%%%%%%%%%%%%%%%%%%%%

\begin{center}
{\Large
PHYS 20323/60323: Fall 2023 - LaTeX Example
}\\
%%%%%%%%%%%%%%%%%%%%%%%%%%%%%%%%%%%%%%%%%%%%%%%%%%%%%%%%%%%%
%{\large Project: Your Name Here}\\\vskip0.25in
%%%%%%%%%%%%%%%%%%%%%%%%%%%%%%%%%%%%%%%%%%%%%%%%%%%%%%%%%%%%
\end{center}
%%%%%%%%%%%%%%%%%%%%%%%%%%%%%%%%%%%%%%%%%%%%%%%%%%%%%%%%%%%%
% Section Heading
%%%%%%%%%%%%%%%%%%%%%%%%%%%%%%%%%%%%%%%%%%%%%%%%%%%%%%%%%%%%
\begin{enumerate}

\item {\bf The following questions refer to stars in the Table below.} \\
Note: There may be multiple answers.

% Table
\begin{tabular}{|l|c|c|c|c|c|}\hline
Name & Mass & Luminosity 3 & Lifetime & Temperature & Radius\\\hline
$\eta$ Car. & 60. \textsl{M}$_\odot$ & $10^{6}$ \textsl{L}$_\odot$ & $8.0 \times 10^{5}$ years &  &  \\\hline
$\epsilon$ Eri. & 6.0 \textsl{M}$_\odot$ & $10^{3}$ \textsl{L}$_\odot$ &  & 20,000 K &   \\\hline
$\delta$ Scu. & 2.0 \textsl{M}$_\odot$ &  & $5.0 \times 10^{8}$ years &  & 2 R$_\odot$ \\\hline
$\beta$ Cyg. & 1.3 \textsl{M}$_\odot$ & $3.5$ \textsl{L}$_\odot$ &  &  &   \\\hline
$\alpha$ Cen. & 1.0 \textsl{M}$_\odot$ &  &  &  & 1 R$_\odot$ \\\hline
$\gamma$ Del. & 0.7 \textsl{M}$_\odot$ &  & $4.5 \times 10^{10}$ years & 5000 K &   \\\hline
\end{tabular}\vskip 0.2in

\begin{itemize}
\item[(a)] (4 points) Which of these stars will produce a planetary nebula. \\
\item[(b)] (4 points) Elements heavier than Carbon will be produced in which stars. \\
\end{itemize}



\item An electron is found to be in the spin state (in the \textit{z}-basis): $\chi=$A\begin{pmatrix} 3\textit{i}\\ 4 \end{pmatrix}

\begin{itemize}
\item[(a)] (5 points) Determine the possible values of \textsl{A} such that the state is normalized. \\
\item[(b)] (5 points) Find the expectation values of the operators {\color{red}$S_x$}, {\color{purple}$S_y$}, {\color{orange}$S_z$} and $\vec{S}^{2}$. \\
\end{itemize}

The matrix representations in the \textit{z}-basis for the components of electron spin operators are
given by: \\\\
{\color{red}\textbf{S$_x$}$ = $\scriptsize{$\dfrac{\hbar}{2}$}\normalsize{\begin{pmatrix} 0 & 1\\ 1 & 0 \end{pmatrix}}} {\color{purple};\quad\quad\quad\quad\textbf{S$_y$}$ = $\scriptsize{$\dfrac{\hbar}{2}$}\normalsize{\begin{pmatrix} 0 & -\textit{i}\\ \textit{i} & 0 \end{pmatrix}}} {\color{orange};\quad\quad\quad\quad\textbf{S$_z$}$ = $\scriptsize{$\dfrac{\hbar}{2}$}\normalsize{\begin{pmatrix} 1 & 0\\ 0 & -1 \end{pmatrix}}} \\



\item The average electrostatic field in the earth’s atmosphere in fair weather is approximately given: \\
\begin{equation}
\textit{$\vec{E}$} = \textit{E}_0\Bigl(\textit{Ae}^{-\alpha z}+\textit{Be}^{-\beta z}\Bigr)\hat{z},\\
\end{equation}

where A,B,$\alpha$,$\beta$ are positive constants and \textit{z} is the height above the (locally flat) earth surface.

\begin{itemize}
\item[(a)] (5 points) Find the average charge density in the atmosphere as a function of height \\
\item[(b)] (5 points) Find the electric potential as a function height above the earth. \\
\end{itemize}

\end{enumerate}
\end{document}